\PassOptionsToPackage{pdfpagelabels=false,unicode}{hyperref}
\documentclass[10pt, a4paper, sans]{moderncv}

% Set PDF metadata using \hypersetup.
\usepackage[unicode, pdflang={en-GB}]{hyperref}
\hypersetup{
  pdftitle={Subhransu S. Bhattacharjee's Résumé},
  pdfauthor={Subhransu S. Bhattacharjee},
  pdfsubject={Résumé},
  pdfkeywords={AI, Machine Learning, Optimization, 3D Vision, Quantitative Research},
}

\usepackage[T1]{fontenc}
\usepackage{multicol}
\usepackage{amsmath}
\usepackage{verbatim}
\usepackage[hmargin=1.25cm, bottom=1in, top=1in]{geometry}
\usepackage{comment}
\usepackage{float, multirow}

\usepackage{moderncvcompatibility}
\moderncvstyle{banking}
\moderncvcolor{black}
\nopagenumbers{}
\renewcommand\footrule{\hrule width\textwidth}
\usepackage{enumitem,url}
\usepackage{array,adjustbox, xcolor,fancyhdr}
\usepackage{etoolbox}

% Document Header
\title{\huge Résumé}
\name{\huge Subhransu S.}{\huge Bhattacharjee}
\address{Skaidrite Darius Building, The Australian National University, Acton, ACT 2601}
\newcommand*{\gmail}{\href{mailto:1ssb.rudra@gmail.com}{1ssb.rudra@gmail.com}}
\email{Subhransu.Bhattacharjee@anu.edu.au}

% Needs a space
\mobile {+61474224742}
\homepage {1ssb.github.io}

% Redefine custom social commands with new names
\newcommand*{\githublink}{\href{https://github.com/1ssb}{GitHub}}
\newcommand*{\linkedinlink}{\href{https://www.linkedin.com/in/1ssb/}{LinkedIn}}
\newcommand*{\googlescholarlink}{\href{https://scholar.google.com/citations?hl=en&user=Ngk4emsAAAAJ&view_op=list_works&sortby=pubdate}{Google Scholar}}

% Add social information
\extrainfo{
    \faEnvelope~\href{mailto:1ssb.rudra@gmail.com}{1ssb.rudra@gmail.com} ~~~
    \faGithub~\githublink ~~~
    \faLinkedin~\linkedinlink ~~~
    \aiGoogleScholar~\googlescholarlink
}
\setlength{\emergencystretch}{3em}
\tolerance=1000

% Document
\begin{document}
\makecvtitle
\vspace{-2.5em} 
\section{Education}
\cventry{April 2023-Present}{Research School of Computing, Australian National University}{Doctor of Philosophy in Artificial Intelligence}{Ongoing}{Australia}{
  \textbullet Supervisors: Dr.\ Rahul Shome, Dr.\ Dylan Campbell \& Prof.\ Stephen Gould\\
  \textbullet Specializations: Vision Language Models, Non-Convex Optimization, Diffusion Models \& 3D Computer Vision\\
  \textbullet Attended: Robotic Vision Summer School, 2024; Optiver PhD Quant Lab Program, 2024 (Acceptance Rate 6\%)\\
  \textbullet Participated by invitation to the all-paid PhD summit by Citadel \& Citadel Securities, London, 2025 (Acceptance Rate 4.8\%)\\
  \textbullet Thesis Topic: \textit{A Probabilistic 3D Spatio-Semantic Reasoning Framework using Generative Models}\\
  \textbf{Courses Audited:} \textit{Task \& Motion Planning in Robotics, Convex Optimization, Differential Geometry \& Probability Theory}
}
\vspace{0.5em}
\cventry{July 2018 - Dec 2022}{College of Systems \& Society, Australian National University}{Bachelor of Engineering}{First Class, Honours}{Australia}{
  \textbullet Major in Mechatronic Systems Engineering (Graduated cum laude in Honors' cohort)\\
  \textbullet Minors in Mathematics \& Electronic Communication Systems\\
  \textbullet Summer School at the \textbf{London School of Economics}, 2019: Practical Machine Learning: Grade-A\\
  \textbullet \href{https://coursera.org/share/393423005d9dec9f00e3c0895c54868f}{\textcolor{black}{Online Certification in Game Theory, \textbf{Stanford University}}}\\
  \textbullet Thesis Project: \href{https://arxiv.org/pdf/2203.02140}{Whiplash Gradient Descent Dynamics} (Supervisor: Professor Ian Petersen).\\
  \textbullet Transferred to ANU from VIT, India in 2020 (Top 1\% of branch).\\
  \textbf{Courses Audited:} \textit{Non-linear Control Theory, Network Optimization \& Control, Information Theory, Mathematical Analysis}\\
}
\vspace{-1em}
\section{Scholarships \& Awards}
\begin{enumerate}\itemsep0em
    \item \textbf{2023:} ANU International University Research Scholarship with HDR Merit Stipend (Acceptance Rate: 2.18\%)
    \item \textbf{2022:} Course Highest in Robotics (ENGN4627)
    \item \textbf{2022:} Highly recommended paper in the Asian Control Conference
    \item \textbf{2021:} High commendation award in the Australia and New Zealand Control Conference
    \item \textbf{2020:} ANU CECS Undergraduate International Scholarship and Partner Institute scholarship -- $50\%$ tuition scholarship.
\end{enumerate}
\vspace{-1em}
\section{Publications}
\cvitem{Subhransu S. Bhattacharjee$^*$, Dylan Campbell \& Rahul Shome}{Believing is Seeing: Unobserved Object Detection using Generative Models, \href{https://arxiv.org/abs/2410.05869}{IEEE/CVF Computer Vision and Pattern Recognition, 2025}}
\cvitem{Subhransu S. Bhattacharjee$^*$ \& Ian Petersen}{Analysis of the Whiplash Gradient Descent Dynamics,\\
DOI: \href{https://onlinelibrary.wiley.com/doi/10.1002/asjc.3153}{10.1002/asjc.3153}, Asian Journal of Control, Special Edition, 2023}
\cvitem{Subhransu S. Bhattacharjee$^*$ \& Ian Petersen}{Analysis of closed-loop inertial gradient dynamics,\\ DOI:\href{https://ieeexplore.ieee.org/document/9828104}{10.23919/ASCC56756.2022.9828104}, Asian Control Conference, 2022}
\cvitem{Subhransu Bhattacharjee$^*$ \& Ian Petersen}{A closed loop gradient descent algorithm applied to Rosenbrock's function, DOI:\href{https://ieeexplore.ieee.org/document/9628258}{10.1109/ANZCC53563.2021.9628258}, Australia and New Zealand Control Conference, 2021}
%
\section{Experience}
\cventry{March 2025 -- Present}{Casual Sessional Academic --- Cybernetics}{School of Cybernetics, Australian National University}{}{\textit{Employers: Dr.\ Safiya Okai-Ugbaje}}{%
\begin{itemize}[itemsep=0em]
  \item Tutoring hands-on laboratory sessions for Masters students.
  \item Developing projects on Microprocessors, Robotics and Machine Learning.
  \item Teaching introductory programming using Python.
\end{itemize}}
%
\cventry{Nov 2024 -- Feb 2025}{Quantitative Research Intern}{Optiver APAC, Sydney}{}{}{%
\begin{itemize}[itemsep=0em]
  \item Developed, implemented \& back-tested a statistical arbitration model on the Hong Kong exchange.
  \item Performed statistical analyses on large-scale financial data to uncover market patterns and operational inefficiencies in Korean market.
  \item Collaborated with traders and developers to build a proprietary, real-time Machine Learning decision-making system.
\end{itemize}}
%
\newpage
\cventry{Sep 2023 -- Sep 2024}{Graduate Research Assistant --- Fintech \& AI}{Research School of Management, Australian National University}{}{\textit{Principal Investigator: Dr. Priya Muthukannan}}{%
\begin{itemize}[itemsep=0em]
  \item Conducted qualitative analyses of open banking regimes using dynamic capabilities frameworks.
  \item Delivered introductory courses in data analysis for Business Information Systems.
  \item Developed innovative frameworks to assess the impact of AI on banking responses to technological shifts.
\end{itemize}}
%
\cventry{Jul 2022 -- Sep 2023}{Casual Sessional Academic --- Engineering}{School of Engineering, Australian National University}{}{\textit{Employers: Prof. Ian Petersen \& Prof. Iman Shames}}{%
\begin{itemize}[itemsep=0em]
  \item Tutored laboratory sessions for Advanced Control Systems (ENGN8824) for a cohort of 12 master’s students.
  \item Facilitated interactive problem-solving sessions for 34 students in Network Optimization and Control (ENGN4628).
  \item Led focused tutoring sessions for 16 students in Power Systems and Electronics (ENGN4625).
\end{itemize}}
%
\cventry{Mar 2022 -- Jun 2022}{Undergraduate Researcher --- Foundational Deep Learning}{School of Computing, Australian National University}{}{\textit{Supervisor: Prof. Richard Hartley, FAA}}{%
\begin{itemize}[itemsep=0em]
  \item Applied neural networks to assess the invertibility of differentiable functions in non-linear processes, achieving a 72\% RMSE hit rate using positional encoding.
  \item Demonstrated the limitations of normalizing flow networks for global invertibility, underscoring neural networks’ limitations as local approximators for smooth functions.
  \item Developed an FPGA-based TIMER algorithm to quantify the computational effort required for minimum convergence, accounting for floating-point precision constraints.
\end{itemize}}
%
\cventry{Dec 2021 -- Mar 2022}{Undergraduate Researcher --- Control \& Optimisation}{Research School of Engineering, Australian National University}{}{\textit{Supervisor: Prof. Ian Petersen, FAA}}{%
\begin{itemize}
  \item Developed a deterministic algorithm that outperformed classical Nesterov-like methods for convex functions.
  \item Applied control theory to design universal Lyapunov-based methods for predicting convergence rates in high-resolution ODE models.
\end{itemize}}
%
\cventry{Mar -- Aug 2021}{Head Automation Intern --- Power Systems Automation}{Calcutta Electric Supply Corporation, India}{}{\textit{Supervisor: Mr. Arindam Sanyal, Director}}{%
\begin{itemize}[itemsep=0em]
  \item Led a cross-functional team of 17 (including 5 interns and 12 field workers) to design and implement a self-healing mechanism for the Ring Main Unit-based power system at Chitpur Hospital Substation during the second COVID-19 wave in India.
\end{itemize}}
%
\cventry{Dec 2020 -- Mar 2021}{ML Research Intern --- Financial NLP}{Decimal Point Analytics, India}{}{\textit{Supervisor: Mr. Paresh Sharma, MD}}{%
\begin{itemize}[itemsep=0em]
  \item Engineered and optimized a financial metadata database for a RoBERTa-based question-answering system, enhancing both accuracy and efficiency.
  \item Conducted client meetings for product reassessment and quality assurance, collaborating with diverse stakeholders to implement performance enhancements.
\end{itemize}}
%
\cventry{Sep -- Nov 2020}{Research Intern}{Laxmi Vilas Bank, India}{}{\textit{Supervisor: Mr. Parthasarathi Mukherjee, Ex-CEO}}{%
\begin{itemize}
  \item Devised a portfolio optimization method using competitive neural networks to analyze multiple time series datasets.
\end{itemize}}
%
\cventry{May -- Aug 2020}{Summer Research Trainee --- Passive Radar Signal Processing}{Armament Research and Development Establishment, DRDO, India}{}{}{%
\begin{itemize}[itemsep=0em]
  \item Developed a Kalman filter–based technique to rapidly select optimal matched filters for incoming radar signals via ambiguity functions, enabling fine-tuning of radar bursts to reduce signal uncertainty.
  \item Implemented an FPGA-based multi-processor interface for real-time analysis of long-range, noise-affected radar signals.
\end{itemize}}
%
\section*{Skills}
\begin{itemize}[leftmargin=*, itemsep=0em]
    \item \textbf{Programming Languages \& Scripting:} Python, C, C++, PyCUDA, R, Embedded C, MATLAB, HTML, CSS, JavaScript
    \item \textbf{ML libraries:} PyTorch, Scikit-learn, Pandas, NumPy, Numba, SciPy, Seaborn, Matplotlib, Jupyter, OpenCV, OpenCL, OpenGL, Transformers, CuPy, XGBoost, LightGBM, Dask, W\&B, Optuna, NLTK
    \item \textbf{Additional Tools \& Frameworks:} Git/GitHub, PySpark, SQL, STM32Cube, Vivado, Simulink, LTSpice, SCADA, DigiSim, SLURM, Blender, SAS, Tableau, \LaTeX, Shell Script
\end{itemize}
\vspace{-1em}
\section*{Certifications \& Services}
\begin{itemize}[leftmargin=*, itemsep=0em]
    \item \textbf{Certifications:} ANU Tutor (Teaching) Program; ANU Lab Training for Researchers; ANU AI in Research \& Teaching; \href{https://coursera.org/share/8204c8b2ca3d84b7fccec7d44edaa8b7}{\textcolor{black}{Machine Learning Production, Deeplearning.AI}}; \href{https://www.coursera.org/account/accomplishments/certificate/JZCYF4MAB3CX}{\textcolor{black}{Project Management, Google}};
    \href{https://www.coursera.org/account/accomplishments/verify/X6R6DZN8VGXN?utm_source=link&utm_medium=certificate&utm_content=cert_image&utm_campaign=pdf_header_button&utm_product=course}{\textcolor{black}{Financial Markets, Yale}};
    \href{https://www.udemy.com/certificate/UC-36c9fc3c-e20c-45fb-994e-776393cdb4ce/}{\textcolor{black}{SCADA, Udemy}}
    \item \textbf{Paper Reviewer:} IROS, 2025; ICRA, 2025; Asian Journal of Control, 2023; Asian Control Conference, 2022; ANZCC, 2021
    \item \textbf{Volunteering Experience:} Thread Together, Sydney (2025), ANU Techlauncher Program (2024); Set4ANU Mentoring (2023--24); Course Representative, ANU (2021); AIESEC VIT (2018--19); Friends of Tribal Society, India (2017--18)
\end{itemize}
\end{document}